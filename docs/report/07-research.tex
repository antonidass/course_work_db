\chapter{Исследовательская часть}

В данном разделе проводится эксперимент, в котором сравниваются временные характеристики получения данных из таблицы прогнозов с помощью GET запроса из базы данных с использованим механизма кэширования и без использования механизма  кэширования.

Цель эксперимента --- сравнить время получения данных с помощью GET запроса о всех прогназах компании с использованием кэширования и без использования кэширования.



\section{Описание эксперимента}
Для сравнения времени запросов с использованием кэширования и без использования кэширования будут выполнены следующие действия:
\begin{enumerate}
	\item Замерим время выполнения запроса к базе данных, отключив механизм кэширования. При этом количество прогнозов в базе данных равняется $n$ элементам.
 	\item Замерим время выполнения запроса к базе данных, включив механизм кэширования. При этом количество прогнозов в базе данных равняется $n$ элементам.
  	\item Повторим пункты $1.$ и $2.$ для n = 10, 50, 100, 500, 1000.
\end{enumerate}

\section{Результат эксперимента}

В таблице \ref{tbl:experiment1} представлены результаты проведенного эксперимента. 

\begin{table}[H]
	\centering
	\caption{Результаты сравнения времени выполнения запроса при включенном и отключенном механизме кэширования}
	\label{tbl:experiment1}
	\resizebox{\textwidth}{!}{%
		\begin{tabular}{|l|l|l|}
			\hline
			\textbf{\begin{tabular}[c]{@{}c@{}}Количество \\ прогнозов\end{tabular}} & \textbf{Время без кэширования, мс} & \textbf{Время с кэшированием, мс} \\ \hline
			10 & 43.643 & 7.72 \\ \hline
			50 & 203.153 & 8.41 \\ \hline
			100 & 310.921 & 7.84  \\ \hline
			500 & 1636.152 & 9.26 \\ \hline
			1000 & 3152.685 & 10.63 \\ \hline
		\end{tabular}%
	}
\end{table}

\section*{Вывод}

На основе результата сравнения времени запроса при включенном и отключенном механизме кэширования можно сделать вывод о том, что приложение с кэшированием данных работает приблизительно в 40 раз быстрее. Данный результат обусловлен тем, что база данных с кэшем уже хранит данные для ответа на запрос. За счет того, что асимптотическая сложность поиска по индексу равна $O(1)$ при любом количестве элементов будет получен одинаковый результат замеренного времени. Однако такой результат можно получить только тогда, когда выборочные данные находятся в кэше, что является не всегда выполнимым условием. 