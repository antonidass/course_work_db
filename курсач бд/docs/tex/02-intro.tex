\chapter*{Введение}
\addcontentsline{toc}{chapter}{Введение}

В настоящее время количество частных инвесторов растет с каждым годом, так, согласно исследованиям \cite{moex} с начала 2021 года число физических лиц, имеющих брокерские счета на Московской бирже увеличилось на 6.2 миллиона человек и достигло 15 миллионов. В большинстве случаев выбор того или иного финансового инструмента у начинающего инвестора зависит от мнений авторитетных аналитиков. Как следствие, возникает необходимость ознакомления с большим количеством различных точек зрений. Ввиду отсутствия должного опыта пользователь вынужден посещать обширное количество интернет-ресурсов. Решением данной проблемы может послужить агрегация множества прогнозов в единое веб-приложение, которое предоставит информацию конечному пользователю в удобном и доступном виде.

Цель работы --- реализовать программное обеспечение для добавления, хранения, редактирования и удаления данных о прогнозах на курсы акций.

Чтобы достигнуть поставленной цели, требуется решить следующие задачи:

\begin{itemize}
    \item формализовать задание и определить необходимый функционал;
    \item проанализировать варианты представления данных и выбрать подходящий вариант для решения задачи;
    \item проанализировать системы управления базами данных и выбрать подходящую систему для хранения данных;
    \item спроектировать базу данных, описать ее сущности и связи;
    \item разработать программный интерфейс для работы с базой данных;
    \item реализовать программное обеспечение, которое позволит получить доступ к данным по средствам REST API \cite{rest-api}.
\end{itemize}
